\chapter*{\centering\Large{Cấu trúc đồ án}}
\addcontentsline{toc}{chapter}{Cấu trúc đồ án}

\begin{tabular}{| p{.2\textwidth} | p{.7\textwidth} |}
\hline
\textbf{Chương 1} & Giới thiệu tổng quan về bài toán VQA, lý do chọn đề tài VizWiz-VQA, 
mục tiêu và phạm vi nghiên cứu, cũng như ý nghĩa thực tiễn của việc giải quyết câu hỏi Yes/No cho người khiếm thị. \\
\hline
\textbf{Chương 2} & Trình bày cơ sở lý thuyết về VQA, các kiến trúc học sâu liên quan, đặc biệt là mô hình CLIP. 
Đồng thời tổng hợp và phân tích các công trình nghiên cứu trước đây về VQA và những thách thức còn tồn tại. \\
\hline
\textbf{Chương 3} & Trình bày phương pháp nghiên cứu: quá trình xử lý bộ dữ liệu VizWiz-VQA cho dạng câu hỏi Yes/No, 
kỹ thuật tiền xử lý dữ liệu và gán nhãn. Mô tả chi tiết phương pháp trích xuất đặc trưng từ CLIP 
và cách xây dựng mô hình phân loại Yes/No. \\
\hline
\textbf{Chương 4} & Cài đặt và thực nghiệm: mô tả thiết lập môi trường, các độ đo đánh giá, 
kết quả thử nghiệm và so sánh hiệu quả giữa các phương pháp. \\
\hline
\textbf{Chương 5} & Thảo luận, nêu ra khó khăn và thách thức khi áp dụng CLIP cho VizWiz-VQA, 
rút ra kết luận từ kết quả thực nghiệm và đề xuất hướng phát triển trong tương lai. \\
\hline
\end{tabular}
