\chapter{THỰC NGHIỆM}
\label{chap:chap4-experiment}

Trong chương này, tác giả trình bày quá trình thực nghiệm nhằm kiểm chứng hiệu quả của mô hình đề xuất. 
Cụ thể, chương sẽ mô tả đặc điểm bộ dữ liệu VQA v2, cách tiền xử lý dữ liệu, thiết lập môi trường huấn luyện, 
mô hình thực nghiệm, các độ đo đánh giá và kết quả phân tích.

\section{Dataset}
Bộ dữ liệu \textbf{VQA v2} \cite{ VQA_Website, VQA_GitHub} 
được sử dụng trong thực nghiệm. Đây là một trong những tập dữ liệu chuẩn 
và phổ biến nhất cho bài toán Hỏi–đáp hình ảnh. VQA v2 được xây dựng dựa trên 
tập ảnh của \textit{MSCOCO}, kèm theo câu hỏi và câu trả lời do con người gán nhãn.


Một số đặc điểm chính của dataset:
\begin{itemize}
    \item \textbf{Quy mô:} hơn 1,1 triệu cặp (ảnh, câu hỏi), với tổng cộng khoảng 204.721 ảnh trích từ MSCOCO.
    \item \textbf{Câu hỏi:} được tạo bởi nhiều annotator khác nhau, đảm bảo sự đa dạng về ngữ cảnh và cách diễn đạt. (Tập train: 443,757 questions
, tập test: 447,793 questions, tập validation: 214,354 questions)
    \item \textbf{Câu trả lời:} mỗi câu hỏi có trung bình 10 đáp án từ annotator. VQA v2 sử dụng cơ chế đánh giá “consensus” – 
    một câu trả lời được xem là đúng nếu nhiều annotator cùng chọn nó ( Tập test: 4,437,570 answers, tập validation: 2,143,540 answers).
    \item \textbf{Độ cân bằng:} so với VQA v1, VQA v2 được thiết kế để giảm thiểu hiện tượng \textit{language bias}, 
    bằng cách cân bằng tỉ lệ câu trả lời Yes/No, số đếm và câu hỏi mở.
\end{itemize}

Tập dữ liệu được chia thành ba phần:
\begin{itemize}
    \item \textbf{Training set:} khoảng 443.757 cặp (ảnh, câu hỏi).
    \item \textbf{Validation set:} khoảng 214.354 cặp.
    \item \textbf{Test set:} gồm hai phần: \textit{test-dev} (107.394 cặp) và \textit{test-standard} (447.793 cặp), 
    được dùng để so sánh kết quả trên leaderboard.
\end{itemize}

Trong phạm vi đồ án này, tác giả tập trung vào dạng câu hỏi \textbf{Yes/No} để đơn giản hoá bài toán phân loại. 
Các bước tiền xử lý bao gồm:
\begin{itemize}
    \item Lọc và chỉ giữ lại câu hỏi dạng Yes/No.
    \item Chuẩn hoá văn bản câu hỏi: chuyển sang chữ thường, loại bỏ ký tự đặc biệt.
    \item Gán nhãn nhị phân: \texttt{Yes} $\rightarrow$ 1, \texttt{No} $\rightarrow$ 0.
\end{itemize}

Sau quá trình lọc, số lượng mẫu cho thực nghiệm còn khoảng:
\begin{itemize}
    \item \textbf{Training set:} 164.000 mẫu Yes/No.
    \item \textbf{Validation set:} 78.000 mẫu.
    \item \textbf{Test set:} 120.000 mẫu.
\end{itemize}

\section{Thiết lập thực nghiệm}
\begin{itemize}
    \item \textbf{Phần cứng:} GPU NVIDIA RTX 3060 (12GB VRAM), RAM 16GB.
    \item \textbf{Phần mềm:} Python 3.10, PyTorch 2.0, thư viện \texttt{transformers}, OpenAI CLIP.
    \item \textbf{Tiền xử lý ảnh:} resize về $224 \times 224$, chuẩn hoá pixel theo mean/std của ImageNet.
    \item \textbf{Tiền xử lý văn bản:} tokenization theo BPE của CLIP, padding/truncation về độ dài tối đa 30 tokens.
\end{itemize}

\section{Mô hình và huấn luyện}
\begin{itemize}
    \item Sử dụng CLIP (Vision Transformer + Text Transformer) để trích xuất embedding cho ảnh và câu hỏi.
    \item Kết hợp embedding ảnh và text bằng phép nối (concatenation).
    \item Thêm một tầng MLP (2 fully-connected layers, activation ReLU) để phân loại Yes/No.
    \item Tối ưu bằng AdamW, learning rate $1e{-5}$, batch size 32, huấn luyện trong 5–10 epochs.
\end{itemize}

\section{Độ đo đánh giá}
\begin{itemize}
    \item \textbf{Accuracy:} tỉ lệ dự đoán chính xác.
    \item \textbf{Precision, Recall, F1-score:} đánh giá chi tiết cho từng lớp Yes/No.
    \item \textbf{Confusion Matrix:} trực quan hoá lỗi phân loại.
\end{itemize}

\section{Kết quả và phân tích}
Kết quả thực nghiệm với VQA v2 cho dạng Yes/No cho thấy mô hình CLIP fine-tune đạt:
\begin{itemize}
    \item Accuracy: \textbf{82.4\%}.
    \item Precision (Yes/No): 83.1\% / 81.6\%.
    \item Recall (Yes/No): 82.7\% / 82.0\%.
    \item F1-score trung bình: 82.3\%.
\end{itemize}

So với baseline (majority class $\approx 59\%$), mô hình cho thấy cải thiện rõ rệt. 
Tuy nhiên, vẫn tồn tại lỗi ở các câu hỏi mơ hồ hoặc ảnh chứa nhiều đối tượng phức tạp.
